    \section{1}\label{section}

    \subsection{(a)}\label{a}

\begin{enumerate}
\def\labelenumi{\arabic{enumi}.}
\tightlist
\item
  The selling price: \(400 \times \$25.31 = \$10,124\)
\item
  The buying price: \(400 \times \$22.87 = \$9,148\)
\item
  The profit: \(\$10,124 - \$9,148 = \$976\)
\end{enumerate}

The profit earned is \(\boxed{\$976}\).

    \subsection{(b)}\label{b}

\[\text{Selling Comm.} = 0.3\% \times \text{Selling Price} = 0.003 \times \$10,124 = \$30.372\]

\[\text{Buying Comm.} = 0.3\% \times \text{Buying Price} = 0.003 \times \$9,148 = \$27.444\]

\[\text{Total Comm.} = \text{Selling Comm.} + \text{Buying Comm.} = \$30.372 + \$27.444 = \$57.816\]

\[\text{Profit with Comm.} = \text{Profit without Comm.} - \text{Total Comm.} = \$976 - \$57.816 = \$918.184\]
Rounding to two decimal places, the profit is \(\boxed{918.18}\)

    \subsection{(c)}\label{c}

The initial proceeds from the short sale is \(\$10,124\) (from part a)
The interest lost using the 6-month interest rate of 3\%:
\(3\% \times \$10,124 = 0.03 \times \$10,124 = \$303.72\)

The interest lost during the 6 months is \(\boxed{\$303.72}\).

    \section{2}\label{section}

    \subsection{(a)}\label{a}

    \begin{center}
    \adjustimage{max size={0.9\linewidth}{0.9\paperheight}}{/home/adrian/prosjfin/ex/1/outputs_6_0.png}
    \end{center}
    { \hspace*{\fill} \\}
    
    \begin{center}
    \adjustimage{max size={0.9\linewidth}{0.9\paperheight}}{/home/adrian/prosjfin/ex/1/outputs_6_1.png}
    \end{center}
    { \hspace*{\fill} \\}
    
    \subsection{(b)}\label{b}

We observe that as the strike price of the put options increases from
\$38 to \$46, the premium also rises, from \$1.72 to \$6.10. This is
because higher strike put options offer greater potential payoff. For
any stock price below the strike, a put with a higher strike will always
yield a larger payoff. Consequently, the breakeven stock price, which is
the strike price minus the premium, also increases with the strike
price, indicating a wider range of stock prices where the option can be
profitable. While the maximum possible loss is limited to the premium
paid for all options, the potential profit increases with higher strike
prices when the stock price falls significantly.

The reason for the increasing premiums is primarily due to the higher
intrinsic value and greater probability of higher strike put options
finishing in-the-money. For a given current stock price, a higher strike
put has a higher or equal intrinsic value. Furthermore, it is more
probable that the stock price will fall below a higher strike price like
\$46, compared to a lower one like \$38. This increased likelihood of a
positive payoff, along with the greater potential for profit in a
significant downturn, justifies the higher premium for put options with
higher strike prices. Essentially, investors pay more for increased
downside protection and a higher chance of the option being valuable at
expiration.

    \section{3}\label{section}

    \subsection{(a)}\label{a}

Given that the premiums on the call and put options are approximately
the same and cancel each other out, and they have the same strike price
and time to expiration, the strike price must be close to the
at-the-money level. More precisely, the strike price is likely near the
forward price of the stock for the expiration date of the options. This
is implied by put-call parity: if call and put premiums are equal,
\(C \approx P\), then \(S_0 - Ke^{-rT} \approx 0\), so
\(K \approx F_0\).

\subsection{(b)}\label{b}

The position created by purchasing a call and selling a put option with
the same strike and expiration is a synthetic long stock position, or a
synthetic purchased stock.

\subsection{(c)}\label{c}

For a synthetic purchased stock with zero net premium inclusive of the
bid-ask spread, the strike price will likely be very close to the
forward price. To achieve a zero net premium, the ask price of the call
(cost) must be approximately equal to the bid price of the put
(revenue). This condition is generally met when the strike price is near
the forward price.

\subsection{(d)}\label{d}

Similarly, for a synthetic short stock with zero net premium inclusive
of the bid-ask spread, the strike price will also be very close to the
forward price. To achieve zero net premium, the bid price of the call
(revenue) must be approximately equal to the ask price of the put
(cost), which again occurs when the strike price is near the forward
price.

\subsection{(e)}\label{e}

No, the ``transaction fees'' are not really ``a wash''. The quote is an
oversimplification. ``Transaction fees'' here primarily refer to the
bid-ask spread, which is a real cost. When creating a synthetic stock,
you buy at the ask price and sell at the bid price, inherently incurring
the bid-ask spread. Even if the net premium is zero inclusive of the
bid-ask spread, it just means the initial cash flow is minimal, but the
cost of the bid-ask spread is still present. Furthermore, this statement
ignores other transaction fees like brokerage commissions and exchange
fees, which are additional costs not ``washed away''.

    \section{4}\label{section}

    \subsection{(a)}\label{a}

After constructing the binomial tree for the stock price, we get that
the stock prices at 6 months are: \(S_{uu} = \$56.18\),
\(S_{ud} = S_{du} = \$50.35\), \(S_{dd} = \$45.125\).

We use the formula for risk-neutral probabilities:
\(p = \frac{e^{r\Delta t} - d}{u - d}\), where \(r = 5\%\),
\(\Delta t = 0.25\), \(u = 1.06\), \(d = 0.95\). Calculating, we get
\(p \approx 0.56889\) and \(1-p \approx 0.43111\).

For the call option payoffs at expiration, we have:
\(C_{uu} = \max(S_{uu} - K, 0) = \max(56.18 - 51, 0) = \$5.18\),
\(C_{ud} = C_{du} = \max(S_{ud} - K, 0) = \max(50.35 - 51, 0) = \$0\),
\(C_{dd} = \max(S_{dd} - K, 0) = \max(45.125 - 51, 0) = \$0\).

Working backwards through the tree, we calculate the call option values
at Time 3 months:
\(C_u = e^{-r\Delta t} [p C_{uu} + (1-p) C_{ud}] \approx \$2.9103\),
\(C_d = e^{-r\Delta t} [p C_{ud} + (1-p) C_{dd}] = \$0\).

Finally, the value of the 6-month European call option at Time 0 is:
\(C_0 = e^{-r\Delta t} [p C_u + (1-p) C_d] \approx \$1.6359\).

The value of the 6-month European call option is approximately
\(\boxed{\$1.64}\).

\subsection{(b)}\label{b}

For the put option payoffs at expiration, we have:
\(P_{uu} = \max(K - S_{uu}, 0) = \max(51 - 56.18, 0) = \$0\),
\(P_{ud} = P_{du} = \max(K - S_{ud}, 0) = \max(51 - 50.35, 0) = \$0.65\),
\(P_{dd} = \max(K - S_{dd}, 0) = \max(51 - 45.125, 0) = \$5.875\).

Working backwards through the tree, we calculate the put option values
at Time 3 months:
\(P_u = e^{-r\Delta t} [p P_{uu} + (1-p) P_{ud}] \approx \$0.2767\),
\(P_d = e^{-r\Delta t} [p P_{ud} + (1-p) P_{dd}] \approx \$2.8676\).

Finally, the value of the 6-month European put option at Time 0 is:
\(P_0 = e^{-r\Delta t} [p P_u + (1-p) P_d] \approx \$1.3766\).

The value of the 6-month European put option is approximately
\(\boxed{\$1.38}\).

\subsection{(c)}\label{c}

Using the call-put parity formula: \(C + Ke^{-rT} = P + S_0\), we
calculate
\(Ke^{-rT} = 51 \times e^{-0.05 \times 0.5} \approx \$49.7408\).

Then, we check both sides of the equation:
\(C + Ke^{-rT} = \$1.6359 + \$49.7408 = \$51.3767\)
\(P + S_0 = \$1.3766 + \$50 = \$51.3766\)

Since \(\$51.3767 \approx \$51.3766\), call-put parity is verified.
