    \section{1}\label{section}

\subsection{(a)}\label{a}

\begin{itemize}
\tightlist
\item
  For \(S_T = \$38\): Payoff \(= 48 - 38 = \$10\)
\item
  For \(S_T = \$43\): Payoff \(= 48 - 43 = \$5\)
\item
  For \(S_T = \$48\): Payoff \(= 48 - 48 = \$0\)
\item
  For \(S_T = \$53\): Payoff \(= 48 - 53 = -\$5\)
\item
  For \(S_T = \$58\): Payoff \(= 48 - 58 = -\$10\)
\end{itemize}

The payoffs for the short forward position are
\(\boxed{\$10, \$5, \$0, -\$5, -\$10}\) for spot prices of
\(\boxed{\$38, \$43, \$48, \$53, \$58}\) respectively.

\subsection{(b)}\label{b}

For a long put option with a strike price \(K = \$48\), the payoff at
maturity is given by \(\max(K - S_T, 0)\).

\begin{itemize}
\tightlist
\item
  \textbf{If \(S_T = \$38\):} Payoff =
  \(\max(48 - 38, 0) = \max(10, 0) = \$10\)
\item
  \textbf{If \(S_T = \$43\):} Payoff =
  \(\max(48 - 43, 0) = \max(5, 0) = \$5\)
\item
  \textbf{If \(S_T = \$48\):} Payoff =
  \(\max(48 - 48, 0) = \max(0, 0) = \$0\)
\item
  \textbf{If \(S_T = \$53\):} Payoff =
  \(\max(48 - 53, 0) = \max(-5, 0) = \$0\)
\item
  \textbf{If \(S_T = \$58\):} Payoff =
  \(\max(48 - 58, 0) = \max(-10, 0) = \$0\)
\end{itemize}

\subsection{(c)}\label{c}

Comparing the payoffs, we see that for spot prices at or below \$48, the
payoffs are the same for both the short forward and long put. However,
for spot prices above \$48, the short forward position leads to negative
payoffs (losses), while the long put option payoff is capped at \$0,
preventing further losses below \$0.

The long put option should be more expensive than the short forward
contract (considering the premium for the put option and zero initial
cost for the forward). This is because the long put option provides
downside protection that the short forward does not. The put option
limits losses when the spot price increases above the strike price
(\$48), whereas the short forward position has unlimited potential
losses as the spot price rises. Investors are willing to pay a premium
for the limited downside risk offered by the put option compared to the
unlimited risk exposure of a short forward position beyond the agreed
forward price. Therefore, to acquire the downside protection inherent in
the put option, one must pay a premium, making it ``more expensive'' in
terms of upfront cost or value compared to entering a short forward
contract at no initial cost but with unbounded potential loss.

In summary, the long put is more expensive because it offers limited
downside risk, unlike the short forward position.

    \section{2}\label{section}

\subsection{(a)}\label{a}

The forward curve shows an initial upward trend from December Year 0 to
June Year 1, likely due to the cost of carry (interest and storage)
increasing with time, and potentially expectations of rising spot
prices. The significant drop in September Year 1 could be attributed to
seasonality in widget supply or demand, or a specific market event
expected to lower prices around that time. The curve then resumes an
upward trend from September Year 1 to June Year 3, indicating that
longer-term factors like cost of carry and general price appreciation
expectations dominate again after the short-term anomaly in September.

\subsection{(b)}\label{b}

\begin{enumerate}
\def\labelenumi{\arabic{enumi}.}
\tightlist
\item
  Initial Investment (Year 0 December): Assume spot price
  \(S_0 = \$3.00\).
\item
  Revenue at Close (March Year 1): Forward price
  \(F_{0,0.25} = \$3.075\).
\item
  Storage Cost (at Year 1 March): \(\$0.03\).
\item
  Net Cash Inflow at Year 1 March: \(\$3.075 - \$0.03 = \$3.045\).
\item
  Return over 3 months:
  \(\frac{\$3.045 - \$3.00}{\$3.00} = 0.015 = 1.5\%\).
\item
  Annualized continuously compounded rate of return:
  \(\frac{\ln(1.015)}{0.25} \approx 5.96\%\).
\end{enumerate}

Yes, the annualized rate of return of approximately \(\boxed{5.96\%}\)
is sensible. It is very close to the risk-free interest rate of 6\%,
which is expected in an efficient market for a cash-and-carry strategy,
after accounting for storage costs. The slight difference is likely due
to rounding or minor market frictions not considered.

    \section{3}\label{section}

\subsection{(a)}\label{a}

In this situation, the lease rate for widgets would likely be very close
to zero. Because widgets don't deteriorate and storage is free,
merchants can keep a large supply at no cost. Since they can adjust
production, they can easily replace any lent widgets. Lending doesn't
really cost them anything. With many merchants able to lend, competition
would push the lease rate down to almost zero.

\subsection{(b)}\label{b}

Even with seasonal demand, the lease rate will probably stay low on
average, but it might change a little with the seasons. The basic
reasons for a low rate still apply: cheap storage and flexible
production. During high demand seasons, borrowing widgets to short sell
might increase, so merchants could maybe charge a slightly higher lease
rate then. But, because supply can adjust and storage is cheap, these
changes would probably be small, and the average lease rate would be
near zero.

\subsection{(c)}\label{c}

This changes things a lot. The lease rate will now likely change with
the seasons and be more up and down. When demand is high but production
is fixed, widgets become less available. Merchants lose more by lending
widgets when they could be selling them at higher prices. So, the lease
rate will likely go up a lot during peak demand times. When demand is
low, merchants might be more willing to lend at a lower rate.

The timing of your short sale matters. If you borrow when demand is high
for a short time, the lease rate will be highest. If you borrow for a
longer period that includes a high demand season, the average lease rate
over that time will be higher too. If you borrow only during low demand
times, you'll probably get a lower lease rate.

\subsection{(d)}\label{d}

Here, the lease rate will relate to the production season, and how long
you borrow will affect the rate. When production is high and demand is
steady, there are lots of new widgets. Merchants will be happy to lend
them out at a low lease rate, maybe even close to zero. When production
is low (off-season), widgets become less common in terms of new supply.
Even with steady demand, this limited production can make widgets more
valuable. The lease rate will likely increase when production is low.

If you borrow for a short time during the production off-season, you
will likely pay the highest lease rate. Borrowing for a period that
includes the off-season will mean a higher average rate. Borrowing only
during peak production will get you the lowest rates.

\subsection{(e)}\label{e}

If widgets can't be stored, the lease rate becomes much more unstable
and depends heavily on the immediate balance of supply and demand.
Seasonal effects in demand or production become much stronger. If demand
is higher than current production, lease rates can jump up a lot because
borrowing is the only way to meet needs. If production is higher than
demand, lease rates can drop sharply, maybe even become negative to
encourage widgets to move out. The lease rate will change a lot with any
small changes in demand or production because there's no stored
inventory to smooth things out. It's less likely the average lease rate
will be near zero; it will be more of a key factor in balancing supply
and demand day to day.

    \section{4}\label{section}

\subsection{(a)}\label{a}

To find the no-arbitrage forward price for delivery in 9 months, we can
use the formula \(F_0 = S_0 e^{rT}\). Here, the current spot price
\(S_0\) is \$1100, the continuously compounding risk-free rate \(r\) is
5\% or 0.05, and the time to delivery \(T\) is 9 months, or 0.75 years.
Plugging these values in, we calculate
\(F_0 = 1100 \times e^{(0.05 \times 0.75)}\), which is approximately
\(1100 \times e^{0.0375}\). Evaluating this, we get about
\(1100 \times 1.038208\), resulting in a forward price of approximately
\$1142.03. Therefore, the no-arbitrage forward price for delivery in 9
months is about \$1142.03.

\subsection{(b)}\label{b}

If a customer wishes to enter a short index futures position and you, as
the market-maker, take the opposite long position, you can hedge this
risk by creating a synthetic short forward position. To do this, you
would first short the S\&R Index, effectively selling \$1100 worth of
it. Then, you would take the \$1100 proceeds from this short sale and
invest them at the risk-free rate of 5\% per annum, compounded
continuously, for the 9-month duration. Let's see how this works as a
hedge. At time 0, you enter into a long forward contract, short sell the
S\&R index and receive \$1100, and invest that \$1100 at the risk-free
rate. Now consider the value at delivery in 9 months. Let \(S_T\) be the
spot price at that time. The payoff from your long forward contract will
be \(S_T - 1142.03\). The investment of \$1100 will have grown to
\(1100 \times e^{0.05 \times 0.75}\), which is approximately \$1142.03.
Finally, you need to cover your short index position, which will cost
you \(S_T\). The net value at delivery is then the sum of these:
\((S_T - 1142.03) + 1142.03 - S_T\), which simplifies to 0. This shows
that regardless of the spot price \(S_T\) at delivery, your net position
will have a value of zero, demonstrating a risk-free hedge.

\subsection{(c)}\label{c}

If a customer wishes to enter a long index futures position, and you
take the opposite short position as the market-maker, you can hedge this
short forward position by creating a synthetic long forward. The
strategy here is to buy the S\&R Index, purchasing \$1100 worth, and
finance this purchase by borrowing \$1100 at the risk-free rate of 5\%
per annum, continuously compounded, for 9 months. Let's examine how this
hedges. At time 0, you enter into a short forward contract, buy the S\&R
index costing \$1100, and borrow \$1100 to pay for it. At delivery in 9
months, again let \(S_T\) be the spot price. The payoff from your short
forward contract will be \(1142.03 - S_T\). The value of the S\&R index
you bought is now \(S_T\). And you will need to repay the borrowed
amount, which has grown to \(1100 \times e^{0.05 \times 0.75}\), or
approximately \$1142.03. The net value at delivery is the sum of these:
\((1142.03 - S_T) + S_T - 1142.03\), which also simplifies to 0. This
demonstrates that this strategy also creates a risk-free hedge, with a
net value of zero at delivery regardless of the spot price \(S_T\).

    \section{5}\label{section}

\subsection{(a)}\label{a}

A forward curve for oil where the forward prices decrease as the
maturity increases and are below the spot price is called an inverted
forward curve or a curve in backwardation.

To estimate the lease rate for different maturities, we can use the cost
of carry model and the given forward prices and interest rates. We can
calculate the approximate lease rate for each maturity using the
formula: \(l = r - \frac{1}{T} \ln(F_0 / S_0)\).

For the 3-month maturity, the lease rate is approximately 9.69\% per
annum. For the 6-month maturity, it is about 8.40\% per annum. For the
1-year maturity, it's around 6.42\% per annum. For the 3-year maturity,
approximately 3.45\% per annum. And for the 5-year maturity, it is about
3.77\% per annum.

The lease rate is positive and generally decreases as the maturity
extends, suggesting the market values immediate oil availability more
highly, and this premium diminishes over time.

\subsection{(b)}\label{b}

Several factors could explain this inverted forward curve for oil. One
key reason is a high convenience yield driven by short-term supply
concerns. If there are worries about immediate oil availability due to
geopolitical events or production issues, market participants might
value having physical oil right now, leading to a higher spot price
compared to future delivery prices.

Another reason could be expectations that oil prices will fall in the
future. If the market anticipates increased future supply or decreased
demand, perhaps due to new energy technologies or economic shifts, then
forward prices for later delivery would be lower to reflect these
expectations.

Storage constraints could also play a role. If storing oil for longer
periods becomes more costly or difficult, it might push down
longer-dated forward prices relative to the spot price.

Market sentiment and risk aversion can also contribute. In uncertain
times, holding physical oil might be seen as less risky than holding
longer-term contracts, again pushing spot prices up and forward prices
down. It's likely a combination of these factors that results in the
observed inverted forward curve.

\subsection{(c)}\label{c}

Even if oil forward prices seem out of line, arbitrage can be difficult.
Transaction costs like fees and bid-ask spreads reduce profits. Storage
for physical oil is expensive and complex. Transportation and logistics
add further costs. Arbitrage takes time, and prices can move unfavorably
before it's complete due to market volatility. Counterparty risk in
forward contracts also exists.
